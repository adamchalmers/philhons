=====
Me
=====

There is a norm of decision theory which says to ignore outcomes whose probability is zero. Because this norm mentions a specific probability value (zero), it is the kind of norm where it makes sense to impose a tolerance: zero plus or minus e

Why shouldn't we take infinite precision to be required? Can't be because it is difficult (because often it's very easy, e.g. St Petersburg only requires high school arithmetic to achieve infinite precision).

=====
Mark Colyvan:
=====

Pp. 18--19: I think you're relying too heavily on Nick's own words. When you quote an author, there should be a point --- it shouldn't just be that you can't be bothered putting the point in your own words.

P.19 You mention that bounded utility is arbitrary and lacks motivation. I agree. You note that RNP is not without motivation but you do not take up the arbitrariness issue. There is clearly some arbitrariness here. Of course an epsilon can be chosen to avoid trouble in any case, but the epsilon will need to be hand picked in each case. You need to say something about this obvious arbitrariness -- especially since you've charged bounded utility with being arbitrary.