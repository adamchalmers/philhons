\documentclass{article}
\usepackage{natbib}
\usepackage{amsmath}
\usepackage{setspace}
\title{Group Epistemology Essay}
\author{Adam Chalmers}
\date{\today}
\begin{document}
\frenchspacing
% \doublespacing
\onehalfspacing
\maketitle


\section{Group Epistemology}

Epistemology is the study and philosophy of learning, belief and knowledge. Traditionally, epistemology is centered around the individual \textemdash{} how one person should learn about the world around them, what knowledge a person has, and what justifies a certain level of belief in a proposition. Group epistemology asks these questions not of individual agents, but \textit{group} agents. 

It often makes sense to ask what a group believes. For example, does a board of investors anticipate their company becoming profitable? Does a jury believe a defendent is innocent? What is the consensus opinion of a panel of climate scientists? What does a team of pundits jointly anticipate the Liberal Party's primary vote percentage to be? In each of these cases, we're not interested in the opinions of any individuals in the group. We're interested in the group's overall opinion, because the group itself has some sort of authority over and above the individuals who comprise it.



 \textemdash{} groups comprised of many individuals. 

\bibliographystyle{apa-good}
\bibliography{thesis}
\end{document}
