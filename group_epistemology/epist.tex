\documentclass{article}
\usepackage{natbib}
\usepackage{amsmath}
\usepackage{setspace}
\title{Group Epistemology Essay}
\author{Adam Chalmers}
\date{\today}
\begin{document}
\frenchspacing
% \doublespacing
\onehalfspacing
\maketitle


\section{Group Epistemology}

Epistemology is the study and philosophy of learning, belief and knowledge. Traditionally, epistemology is centered around the individual \textemdash{} how one person should learn about the world around them, what knowledge a person has, and what justifies a certain level of belief in a proposition. Group epistemology asks these questions not of individual agents, but \textit{group} agents. 

It often makes sense to ask what a group believes. For example, does a board of investors anticipate their company becoming profitable? Does a jury believe a defendent is innocent? What is the consensus opinion of a panel of climate scientists? What does a team of pundits jointly anticipate the Liberal Party's primary vote percentage to be? In each of these cases, we're not interested in the opinions of any individuals in the group. We're interested in the group's overall opinion, because the group itself has some sort of authority over and above the individuals who comprise it.

Although group and individual epistemology share the goals and concerns, group epistemology involves several problems which individual epistemology does not face. For example, how do we aggregate the beliefs of many individuals into one group belief? How many individuals in a group are required to possess some piece of evidence before we can safely say the group as a whole possesses it? How do we resolve contradictions between the beliefs of different group members? These issues simply don't present themselves when considering the beliefs, knowledge and reasoning of individuals. This paper aims examines the specific question of \textit{group credence}: how do we determine the credence\footnote{In this paper I will use `belief' to designate a binary propositional attitude (agents do or don't believe a certain proposition) and credences as probability assignments (an agent may assign probability 0.3 to a proposition being true).} a group has in a given proposition?

One intuitive solution to the group credence problem is known as \textit{unweighted linear averaging}. In this scheme, groups should just average every individual's credence in a proposition to get the overall group credence. Unfortunately this approach yields poor results in a number of cases, which I shall examine below. A variety of more complicated procedures have been proposed which aim to succeed where unweighted linear averaging (or other approaches) fail. In this paper, I will examine a recently-published theory, \citet{dietrich2013probabilistic}'s \textit{Probabilistic Opinion Pooling} (hereafter POP) and evaluate its success as a solution to the group credence problem.

\section{Evaluating group credence procedures}

To demonstrate the complications inherent in group credence functions, I will examine scenarios where the unweighted linear averaging rule yields poor outcomes. Later in this paper I will use these same scenarios to evaluate how successful POP is as a group credence procedure.

In order to outline unweighted linear averaging I must first define some terminology. Call a \textit{credence function} a function which maps propositions to probabilities (between zero and one). Given a group \(G\) of \(n\) individuals and a list of their credence functions \(CR = \langle cr_1, cr_2, ..., cr_n\rangle\), unweighted linear averaging says that the credence function for the entire group G is 

\[cr_G(P,CR) = \frac{1}{n}\times\sum_{cr \in CR}cr(P)\]

\noindent 
\citet{russell2015groupthink} outlines a number of properties which are \textit{prima facie} desirable in group credence functions. Unweighted linear averaging has a number of these properties, such as 

\begin{itemize}
	\item \textbf{Anonymity}: if \(CR'\) is a permutation of \(CR\), \(cr_G(P, CR) = cr_G(P, CR')\) (i.e. the identity of each individual doesn't matter, only their credence).
	\item \textbf{Systematicity}: the credence of a specific proposition \(P\) is a pure function of each individual's credence in P, and nothing else \textemdash{} neither credences for other propositions, nor outside information beyond credences will affect the output of \(CR\).
	\item \textbf{Unanimity}: if each individual in the group has the same credence for P, then the group should have that credence too.
	\item \textbf{Continuity}: small changes in the input individual credences should lead to small changes in the output group credence, i.e. \(cr_G(-,CR)\) should be a continuous function.
\end{itemize}

Unweighted linear averaging has \textbf{anonymity} because it treats every individual in the group the exact same way. Anonymity is motivated by considerations of fairness \textemdash{} a group should use information from all its members equally, and not privilege one member above others. There shouldn't be a small minority responsible for the group's decisions, and there certainly shouldn't be a \textit{dictator} individual whose credences determine the entire group's.

The procedure meets \textbf{systematicity} because it calculates the group credence for a proposition using only the individual credences for that proposition (averaging them). Systematicity is desirable because otherwise group members could manipulate the group's decision by introducing other propositions or outside information (sometimes called `irrelevant alternatives')

\textbf{Unanimity} is a natural consequence of using averaging: if all the numbers being averaged are equal, then the average will be that value too. Unanimity has an intuitive appeal \textemdash{} if every member in the group agrees, then there's no dispute to even resolve, and thus no need to consult a group credence function. Unweighted linear averaging's \textbf{continuity} comes from the linearity of averaging generally: if you slightly change one of the numbers being averaged, you'll only change the average a slight amount. This has some intuitive appeal but is less motivated than other properties.

Unfortunately, unweighted linear averaging fails to meet several other important criteria of a good credence aggregation function. Firstly, it fails to \textit{commute with conditionalisation}. As \citet[pg. 1290]{russell2015groupthink} puts it, ``if every individual updates on new information by conditionalizing, the resulting average credence won’t be the same as what they would get from first averaging their unconditional credences in each proposition, and then conditionalizing on the new evidence.'' Unweighted linear averaging doesn't respect conditionalisation \citep[pg. 87]{loewer1985destroying}, which has serious consequences.

If each individual conditionalises on evidence, and then the group credence is recalculated, the group's final credence function won't be the same as its old credence conditionalised. This means the group can now be tricked into buying a Dutch Book \textemdash{} ``a set of bets bought or sold at such prices as to guarantee a net loss'' \citep{hajek2008dutch}. Generally, an agent is considered irrational if they would ever possibly buy a bet which is guaranteed to lose them more money than they paid for it. Therefore, any group whose credences do not commute with conditionalisation opens themselves up to irrational behaviour \textemdash{} certainly not a desirable property of a credence function.

Unweighted linear averaging also fails at what I call the \textbf{Elephant Problem}. 




\bibliographystyle{apa-good}
\bibliography{epist}
\end{document}
