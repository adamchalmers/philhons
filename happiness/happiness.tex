\documentclass{article}
\usepackage{natbib}
\usepackage{amsmath}
\usepackage{setspace}
\usepackage{pbox}
\title{Philosophy of Happiness Essay}
\author{Adam Chalmers}
\date{\today}
\begin{document}
\frenchspacing
% \doublespacing
\onehalfspacing
\maketitle


\section{Welfare beyond humans}

When you or I make a decision involving another person, we frequently attempt to do right by the other party. We often factor the other person's welfare into this decision \textemdash{} will our actions make this person better or worse-off? We may need a theory of welfare in order to make decisions about other people \textemdash{} a theory which explains what a person's best interests are, what makes their life good for them to live.

Scholars have proposed many theories of welfare, but frequently these theories only analyse welfare \textit{for humans}. However, as the dominant species on the planet, humans frequently make decisions which affect non-humans. We often try to act ethically towards animals, or towards the natural environment (for example, when we consider agricultural or forestry policy). This suggests we, as human decision makers, need a theory of welfare which extends to non-humans. How can we make ethical decisions about animals or environments without considering their welfare? 

Furthermore, as science and technology progress, we can expect to encounter previously unknown beings, whether artificial (such as robots or artificial intelligences) or natural (such as aliens or other spacefaring civilisations). Ideally, humans should have a theory of how to consider these creatures interests \textit{before} we encounter them. If we leave the question of alien or robotic welfare until we encounter these beings, we may discover too late that we have already caused irreparable harm.

How are we to formulate a general theory of non-human welfare? In this paper I propose one solution to this problem by extending Daniel Haybron's work. The theory of welfare outlined in \citet{haybron2008pursuit} is eudaimonic and puts self-fulfilment as constituting welfare. His work focuses on happiness, which is critical to fulfilling one's emotional nature, i.e. fulfilling a very important part of one's self. However, he acknowledges other components of one's self (such as one's deliberative-rational self and one's physical self) which are also important to welfare. I propose that welfare for beings without affective or emotional states (such as plants or robots) consists of fulfilling these components of the self which they \textit{do} possess. This general theory of welfare would help us determine the welfare of non-humans and may thus be crucial in acting ethically towards non-human beings.

In the remainder of this section, I will introduce the philosophical notion of welfare and the various ways we use it. In section 2 I will outline Haybron's theory of welfare and identify ways it could be extended. Section 3 briefly identify some types of non-human beings and discusses how to modify Haybron's welfare theory to accommodate them.

\subsection{What is welfare?}

Roughly speaking, a person's welfare is whatever is good for them, in a prudential sense. Welfare is a measure of how good someone's life is, for the person living it. Note that ``good'' is used not in ethical terms (as in ``the good'' or ``morally good''), but in prudential terms. A sadist might live a good life (in the prudential sense) by beating small animals, even though this life would not be good in the \textit{moral} sense. Welfare is roughly similar to \textit{utility} in ethics. 

This explains the meaning of the philosophical term ``welfare'', but not what welfare actually consists of. Many competing accounts of welfare claim to have identified the exact nature of welfare, but there exists no philosophical consensus over which, if any, is correct.

\textit{Objective list} theories claim one's welfare is a result of their acquiring several intrinsically welfare-increasing goods. For example, \citet{gert1998morality} claims welfare consists of consciousness, ability, freedom and pleasure, and Aristotle \nocite{joachim1951nichomachean}, claims objective goods come from considering the characteristic activities of a species (such as reasoning for humans). \textit{Desire-fulfilment} theory says welfare is the result of achieving one's desires, and \textit{mental state theory} says welfare is merely some pleasurable mental state such as happiness. \textit{Hedonist} theories say welfare is a matter of experiencing pleasure and avoiding pain. 

Each theory has its own problems. Objective lists run the risk of including items that some individual doesn't seem to require for welfare, or missing an item that someone \textit{does} require for welfare. Under mental state theory, a person trapped in Nozick's experience machine or the Matrix would feel (and mentally experience) high welfare, but would (according to critics) in fact be living a miserable existence. Desire fulfilment theory implies that someone who desires to eat gravel and succeeds is in fact increasing their welfare, regardless of how irrational or unhealthy this desire is. People are frequently mistaken about their desires, and have wants which turn out to be highly undesirable once satisfied. Hedonist theories have trouble explaining heroin addiction, which appears to be a highly pleasurable yet welfare-reducing experience. 

No objection is a killer knock-down argument; each theory is taken seriously by the philosophical community. This paper puts forward an objective list theory of welfare, simply because I feel such theories are best suited to analysing non-human welfare. Not all beings will have desires, pleasurable feelings or mental states, so I feel the corresponding welfare theories are ill-suited to the task at hand. This paper's primary goal is to defend its own theory of non-human welfare; not to criticise other theories of welfare. As such, my analysis of desire, hedonist and mental state theories will be limited.

My theory of non-human welfare is an extension of Daniel Haybron's self-fulfilment welfare. In \citep[chapter 9]{haybron2008pursuit}, Haybron proposes a eudaimonistic account of welfare which identifies self-fulfilment as the objective welfare good \textemdash{} in particular, fulfilment of one's emotional nature. 

This paper is not a defense of Haybron; rather, I take Haybron as a suitable framework to ground non-human welfare in. Later I will defend my theory against criticism, but I will only engage with criticism of Haybron where it is relevant to my non-human welfare theory.

\section{Haybron's theory of welfare}

The question of welfare is what increases a life's prudential value. According to \citet[pg. 192]{haybron2008pursuit}, ``some sort of nature-fulfillment is intrinsically valuable.'' Fulfilling one's nature increases one's welfare. But what exactly is `nature-fulfilment'? Answering this question requires a brief explanation of Haybron's notion of the self.

By ``self'' Haybron means ``those aspects of us that are important to making us the distinct individuals we are, that are important to understanding who we are,'' such as the way a person ``sees or thinks about her herself, her life, her ideals, her projects and commitments, and her relationships to society and other people'' \citep[pg. 183, 184]{haybron2008pursuit}. One's self is a sub-part of one's nature. This means all facts about one's self are facts about one's nature, but not all facts about one's nature are facts about one's self. For example, my love of fiction is a part of my self (and thus my nature), but my height is a part of my nature and \textit{not} my self. 

Haybron's theory is eudaimonistic, i.e. it claims that welfare consists of human flourishing. Specifically, Haybron claims human flourishing consists of nature-fulfilment, primarily (but not exclusively) of the self-fulfilment kind. There are many aspects to the self (for example, we have ``an emotional aspect and a rational aspect'' \citep[pg. 193]{haybron2008pursuit}), and when we act in accordance with some aspect, we are fulfilling our self.

Haybron is particularly concerned with our emotional nature. ``To have a certain emotional nature is to be disposed characteristically to be happy in certain circumstances and not others,'' he writes (pg. 184). A person's emotional nature may change over time or be affected by social and cultural factors, but it is inherently \textit{part of that person's self}. Under Haybron's theory, happiness has intrinsic prudential value because it results from fulfilling our emotional self. This means happiness is a sign of self-fulfilment and therefore of welfare \citep[pg. 178]{haybron2008pursuit}.

This graphic summarises the Haybron model of welfare:

[TODO: make a picture]

This account of eudaimonia differs from the traditional Aristotelian account in that it is neither perfectionist nor externalist. Both views posit that welfare requires fulfilling one's nature, but on Aristotle's view this consists in ``the proper exercise of our distinctively human capacities'' whereas Haybron's consists in fulfilling the ``arbitrarily idiosyncratic make-up of the individual'' \citep[pg. 193]{haybron2008pursuit}.

\subsection{Welfare beyond happiness}

As hinted by his book's title (``The Pursuit of Unhappiness''), Haybron's account of welfare is centered around happiness. However, Haybron explicitly states that welfare consists of more than just happiness and the fulfilment of one's emotional nature. Where else does Haybron think welfare can come from?

If welfare (in the form of happiness) can come from fulfilling one's emotional self-aspect, it can also come from fulfilling one's rational or deliberative self-aspect. 


% \begin{center}
% \begin{tabular}{ | l | l | l |}
%   \hline
%   \multicolumn{3}{|c|}{Table heading} \\
%   \hline\hline
%     & Column & Header \\ \hline
%   Row 1 & 1 & 0.1 \\ \hline
%   Row 2 & 0.1  & 1 \\ \hline
% \end{tabular}
% \end{center}
% \noindent


\bibliographystyle{apa-good}
\bibliography{happiness}
\end{document}
